\documentclass[10pt,a4paper]{article}
\usepackage[utf8]{inputenc}
\usepackage{amsmath}
\usepackage{amsfonts}
\usepackage{amssymb}
\usepackage{makeidx}
\usepackage{titling}
\usepackage{blindtext}
\usepackage{graphicx}
\usepackage{caption}
\usepackage{scrextend}
\usepackage{subfig}
\usepackage{wrapfig}
\setlength{\intextsep}{0pt}
\usepackage[export]{adjustbox}
\usepackage[a4paper,margin=1in]{geometry}
\author{Erik Rosvall, Viktor Karlsson, Lucas Grönlund \& Marcus Alsterman \\ \textit{ML Project Group 69}}
\title{Machine Learning Advanced course \\ String Subsequence Kernel}



%\setcounter{chapter}{1}
\begin{document}
	\begin{titlingpage}
		\maketitle
		\begin{abstract}
			\noindent
			\begin{addmargin}[4em]{4em}
			Trying to solve text classification creates many issues, mainly feature extraction and handling multi-labeled data that might be unbalanced. This report looks at how a String Subsequence Kernel can solve the mentioned problems in conjunction with a SVM. This report goes through how Lodhi et al. has presented and validated their algorithm with both recursive and approximate implementations. Lastly we present our own incremental improvement of the approximate String Subsequence Kernel by the use of additional halt conditions within the recursion. 			
	\end{addmargin}
		\end{abstract}
	\end{titlingpage}
	
	\section{Introduction}
	%TODO
% - Intro to text classification (multilable)
\subsection*{Text classification}
In the area of machine learning, classification of unseen data is of great interest. For this to be possible feature vectors of the input data is needed which is trivial for some types of data, but not so much for others. Text documents is an example of one such kind. 

One method for feature extraction often used within the field of text classification is the \textit{kernel method}, transforming the data into high dimensional spaces and comparing similarity between data points through their inner product. Using the kernel trick, the explicit transform into this feature space can be avoided, which enables infinite dimensional features spaces. 

The article \textit{Text classification using string kernels} written by Lohdi et. al. presents a new method of performing this feature extraction which Lohdi et al names \textit{string subsequence kernel} (SSK). This proposed kernel is like a natural extension to the two other methods presented as a base line in the article, namely word kernel (WK) and n-gram kernel (NGK). How these kernels differ will be explained in following sections and then compared in the discussion. 

Kernels are however not the only successful type of method in the text classification field, neural networks (CNN's RCNN's ) as well as a method called \textit{fastText} have shown great results. \textbf{THESE MIGHT BE DISCUSSED LATER AS A POSSIBLE COMPETITOR TO THE SSK}.

% - Purpose of ssk report
% - Kernels
% - Explain wk and ngk
% - Other methods (not kernels)

	
	\section{Method}
	% TODO
Implemented naive kernel, recursive, approx, ngk, wk, (fastText?)

Alignment to compare real and approx K

Possible improvement implementation 

Mention comibation of kernels and why we did not find it necessary to investigate it any further. 



	\section{Results}
	
% TODO Fix subset of data w. four classes
\begin{tabular}{ c | c | c | c | }
	 WK  & Precision & Recall & F1   \\ \hline	
	 acq &  0.974 (0.843) & 0.930 (0.768) & 0.951 (0.802) \\ \hline
	 earn &   0.978 (0.989)& 0.972 (0.867)& 0.976 (0.925)  \\ \hline
	 corn &   0.992 (0.833) & 0.867 (0.710) &  0.923 (0.762) \\ \hline
	 crude &   0.946 (0.910) & 0.957 (0.907) &  0.948 (0.904) \\ \hline
	
\end{tabular}
\captionof{table}{Running Word Kernel ten times and averaging the result. Numbers in parenthesis is the reference value.}

\begin{tabular}{ c | c | c | c | c | }
	NGK & k & Precision & Recall & F1   \\ \hline	
	 & 3 & 0.96 & 0.88 & 0.92     \\ 
	acq & 4 & 0.90 & 0.89 &  0.89    \\
	 & 5 & 0.97 & 0.86 & 0.92     \\ \hline
	 & 3 & 0.97 & 0.93 &  0.95    \\ 
	earn & 4 & 0.99 & 0.93 &  0.96    \\ 
	 & 5 & 0.99 & 0.89 &  0.93    \\ \hline
	 & 3 & 1 & 0.87 & 0.93     \\ 
	corn & 4 & 1 & 0.64 & 0.78     \\ 
	 & 5 & 1 & 0.44 &  0.61    \\ \hline
	 & 3 & 0.90 & 0.90 &  0.90    \\ 
	crude & 4 & 0.92 & 0.86 & 0.89     \\ 
	 & 5 & 1 & 0.73 &  0.84    \\ \hline
\end{tabular}
\captionof{table}{Subset of the NGK runs for different length N. We can clearly see that NGK performs well for small N, but considerably worse for larger N (see appendix).}
\begin{tabular}{ c | c | c | c | c | }
	SSK & k, $ \lambda = 0.5 $ & Precision & Recall & F1   \\ \hline	
	& 3 & 0.96 & 0.88 & 0.92     \\ 
	acq & 4 & 0.90 & 0.89 &  0.89    \\
	& 5 & 0.97 & 0.86 & 0.92     \\ \hline
	& 3 & 0.97 & 0.93 &  0.95    \\ 
	earn & 4 & 0.99 & 0.93 &  0.96    \\ 
	& 5 & 0.99 & 0.89 &  0.93    \\ \hline
	& 3 & 1 & 0.87 & 0.93     \\ 
	corn & 4 & 1 & 0.64 & 0.78     \\ 
	& 5 & 1 & 0.44 &  0.61    \\ \hline
	& 3 & 0.90 & 0.90 &  0.90    \\ 
	crude & 4 & 0.92 & 0.86 & 0.89     \\ 
	& 5 & 1 & 0.73 &  0.84    \\ \hline
\end{tabular}
\captionof{table}{String Subsequence Kernel for different length k. Like it's algorithmic cousin the NGK, it performs well for relative small k, but less well for larger k (again, see appendix for all data).}

\begin{tabular}{ c | c | c | c | c | }
	SSK & $ \lambda  $, k = 5& Precision & Recall & F1   \\ \hline	
	& 0.05 & 1 & 0.92 & 0.96     \\ 
	acq & 0.1 & 1& 1 &  1    \\
	& 0.5 & 0.89 & 1 & 0.94     \\ \hline
	& 0.05 & 1 & 0.98 &  0.99    \\ 
	earn & 0.1 & 0.98 & 1 &  0.99    \\ 
	& 0.5 & 1 & 0.90 &  0.95    \\ \hline
	& 0.05 & 1 & 0.87 & 0.93     \\ 
	corn & 0.1 & 1 & 0.87 & 0.93     \\ 
	& 0.5 & 0.94 & 1 &  0.97   \\ \hline
	& 0.05 & 0.90 & 0.75 &  0.82    \\ 
	crude & 0.1 & 1 & 0.91 & 0.95     \\ 
	& 0.5 & 1 & 0.70 &  0.82    \\ \hline
\end{tabular}
\captionof{table}{Varying $ \lambda $ shows that SSK is not that sensitive to values of $ \lambda $. Generally we can see that unless you choose more extreme values, the SSK performs well. }

% TODO Alignment of S vectors
\begin{figure}
	\centering
	\begin{equation}
		\begin{tabular}{c c}
		\null \hfill
			\subfloat[]{\includegraphics[height = 5cm]{../plots/Alignment_scores.pdf}} \hfill & 
			\subfloat[]{\includegraphics[height = 4.4cm]{../plots/Lodhi_alignment_score.png}} \hfill \null
		\end{tabular}
	\end{equation}
	\captionof{figure}{left is ours, right's Lodhi's. Padding is f*cked.}
\end{figure}

% TODO Approximation check (table 8 in report)
\begin{tabular}{ c | c | c | }
	& s = 1000 & s = 3000   \\ \hline
	acq & 0.96 (0.88)& 0.97 (0.85)\\ \hline
	earn & 0.98 (0.97) & 0.98  (0.97) \\ \hline
	ship & 0.43 (0.10) & 0.63  (0.53) \\ \hline
	corn & 0.84 (0.15) & 0.89 (0.65) \\ \hline
\end{tabular}
\captionof{table}{F1 scores for approximate kernel validation. k = 5 and $ \lambda = 0.5 $. Lodhi's results in parenthesis. }

% 1130


% TODO top 10 classes results
\begin{tabular}{ c | c | c | c | c | c | c | c |}
	& WK & NGK & NGK  & NGK  & SSK & SSK& SSK \\ 
	&  & n = 3& n = 4 & n = 5 & n = 3& n = 4 & n = 5 \\ \hline
	earn & 0.98 & 0.98 &  0.98&  0.98 & 0.98 & 0.98 & 0.98 \\ \hline
	acq & 0.97 & 0.95 &  0.95 &  0.96 & 0.95 & 0.95 & 0.95 \\ \hline
	money-fx & 0.79 & 0.77 &  0.79 & 0.77 & 0.77 & 0.8 & 0.78 \\ \hline
	grain & 0.92 & 0.81 &  0.83& 0.82 & 0.84 & 0.86 & 0.8 \\ \hline
	crude & 0.87 & 0.84 &  0.85 & 0.8 & 0.82 & 0.79 & 0.73 \\ \hline
	trade & 0.8 & 0.73 &  0.77 & 0.77 & 0.72 & 0.79 & 0.77 \\ \hline
	interest & 0.8 & 0.72 &  0.73 & 0.75 & 0.79 & 0.8 & 0.77 \\ \hline
	ship & 0.78 & 0.66 &  0.55 & 0.47 & 0.69 & 0.5 & 0.34 \\ \hline
	wheat & 0.79 & 0.86 &  0.84 & 0.8 & 0.8 & 0.8 & 0.76 \\ \hline
	corn & 0.86 & 0.73 &  0.85 & 0.65 & 0.78 & 0.72 & 0.6 \\ \hline	
\end{tabular}
\captionof{table}{Lodhi's top ten performing classes for the full dataset with an approximating kernel.}
	
	\section{Discussion}
	%TODO Consequences from test runs
We can clearly see the same pattern between our data and that of Lodhi et al, that for higher $ n $ we have a clear drop off in performance. We had to limit our computations because of time limitations, but the general trend is there. Performance quickly drops of for $ n>6 $ for almost all classes. This was consistent across all configurations of the algorithm and the approximated version. Interestingly, the results for SSK isn't super promising, we usually achieve tangential performance with either NGK and WK. This becomes painfully clear when we run the full dataset, where WK outperformed NGK and SSK on most categories at a fraction of the computation time, and that is with our approximation kernel exported to C++ and WK running completely in Python. As Lodhi summarized the findings in his report, that for small text strings and small $ n $ will the SSK perform well, but quickly losses ground. This can be very logical, since when we get more data noise to signal ratio for words goes down and we can rely more on the words rather than some sequence of characters. More documents also makes the $ tfidf $ transform better SOURCE. At the same time will stuff like n-grams and the SSK hit more noisy information since more and longer text string are bound to hit more possible combinations. Another reason for why WK works well compared to SSK might be because of the nature of the dataset. Reuters journalists are proofread and held to a higher linguistic standard than say a chat log from some messenger program. The relative lower amount of spelling errors and stylistic stringency should have a positive effect on WK's performance relative SSK.
\\
As with Lodhi et al's report, we find that SSK is not particularly sensitive to values on the decay parameter $ \lambda $, and as long as one avoids values close to one or zero the performance is generally very similar between classes. What however is very sensitive is the SVM for the unbalanced data that Reuters have. The disparity between classes are big, sometimes thousands to one. We've tried to solve this problem with a variety of technique, but mostly trying to balance the data so that the SVM can find decision boundary that generalizes well. What we've found though is that we didn't manage to find appropriate parameters for larger $ n $. Both NGK and SSK suffered at higher values and we could not replicate the performance mentioned by Lodhi et al.  

%TODO Similarity to fastText and CNN and differences
Looking at how the String Subsequence Kernel behaves in comparison to for example Convoluted Neural Networks and Facebook's fastText algorithm. These both perform very well in different ways. CNN's has been shown to be able to reach very high level of performance, but takes long time to train and can be difficult to configure. fastText is very fast, training times in seconds rather than hours or days, but requires language specific understanding to work. fastText has been shown to be similiarly accurate to CNNs, but at a fraction of the computation time. fastText uses a linear classifier in its base, but uses very large, precomputed, feature vectors and several computational tricks to speed up training significantly.
	
	\bibliographystyle{plain}
	\bibliography{citations.bib}
	
	%\section*{Appendix}
	%\begin{tabular}{| c | c | c | c | c | }
	\hline SSK & $ \lambda  $, k = 5& Precision & Recall & F1   \\ \hline	
	
	& 0.01 & 0.96 & 0.92 & 0.94     \\ 
	& 0.05 & 1 & 0.92 & 0.96     \\ 
	acq & 0.1 & 1& 1 &  1    \\
	& 0.5 & 0.89 & 1 & 0.94     \\ 
	& 0.07 & 0.93 & 0.96 & 0.94     \\ 
	& 0.9 & 0.64 & 0.96 & 0.77     \\ \hline
	
	
	& 0.01 & 0.95 & 0.95 &  0.95    \\	
	& 0.05 & 1 & 0.98 &  0.99    \\ 
	earn & 0.1 & 0.98 & 1 &  0.99    \\ 
	& 0.5 & 1 & 0.90 &  0.95    \\ 
	& 0.7 & 1 & 0.90 &  0.95    \\
	& 0.9 & 0.95 & 0.88 &  0.91    \\\hline
	
	
	
	& 0.01 & 1 & 0.80 & 0.89     \\ 
	& 0.05 & 1 & 0.87 & 0.93     \\ 
	corn & 0.1 & 1 & 0.87 & 0.93     \\ 
	& 0.5 & 0.94 & 1 &  0.97   \\ 
	& 0.7 & 1 & 0.80 & 0.89     \\ 
	& 0.9 & 0.64 & 0.60 & 0.62     \\ \hline
	
	
	& 0.01 & 1 & 0.70 &  0.82    \\
	& 0.05 & 0.90 & 0.75 &  0.82    \\ 
	crude & 0.1 & 1 & 0.91 & 0.95     \\ 
	& 0.5 & 1 & 0.70 &  0.82    \\ 
	& 0.7 & 0.90 & 0.90 &  0.90    \\
	& 0.9 & 0.39 & 1 &  0.56    \\\hline
	
	
\end{tabular}



\begin{tabular}{| c | c | c | c | c | }
	\hline NGK & k & Precision & Recall & F1   \\ \hline
			
	& 3 & 0.96 & 0.88 & 0.92     \\ 
	 & 4 & 0.90 & 0.89 &  0.89    \\
acq	& 5 & 0.97 & 0.86 & 0.92     \\ 
	& 6 & 0.99 & 0.82 & 0.89     \\
	& 7 & 0.98 & 0.79 & 0.87     \\
	& 8 & 1 & 0.67 & 0.80     \\
	\hline
	
	
	
	& 3 & 0.97 & 0.93 &  0.95    \\ 
	 & 4 & 0.99 & 0.93 &  0.96    \\ 
	earn & 5 & 0.99 & 0.89 &  0.93    \\ 
	& 6 & 0.99 & 0.89 &  0.93    \\ 
	& 7 & 0.99 & 0.89 &  0.93    \\ 
	& 8 & 0.99 & 0.89 &  0.93    \\ \hline
	
	
	
	& 3 & 1 & 0.87 & 0.93     \\ 
	 & 4 & 1 & 0.64 & 0.78     \\ 
	corn	& 5 & 1 & 0.44 &  0.61    \\ 
	& 6 & 1 & 0.42 & 0.59     \\ 
	& 7 & 1 & 0.24 & 0.38    \\ 
	& 8 & 1 & 0.28& 0.42     \\ \hline
	
	
	& 3 & 0.90 & 0.90 &  0.90    \\ 
	 & 4 & 0.92 & 0.86 & 0.89     \\ 
	crude & 5 & 1 & 0.73 &  0.84    \\ 
	& 6 & 1 & 0.66 &  0.79    \\
	& 7 & 1 & 0.60 &  0.72    \\
	& 8 & 1 & 0.28 &  0.44    \\ \hline
	
	
	
\end{tabular}
\end{document}